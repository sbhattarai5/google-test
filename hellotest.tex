\section{Hello, Test!}
Let's write a very simple C++ program that uses Google Test.
\subsection{Initializing}
First thing is to include the Google testing tool and initialize it in
the main function. Following should be your template for all the
programs:
\begin{Verbatim}[frame=single]
#include <gtest/gtest.h>                // including the library
  
int main(int argc, char **argv)
{
    testing::InitGoogleTest(&argc, argv); // initializing it
    return RUN_ALL_TESTS();               // this will run all your tests
}
\end{Verbatim}

Compile this using:
\begin{Verbatim}[frame=single]
g++ test.cpp -lgtest -lgtest_main -lpthread
\end{Verbatim}
Let's add a \verb!factorial! function to the code.
\begin{Verbatim}[frame=single]
#include <gtest/gtest.h>                // including the library

int factorial(int n)
{
    if (n < 0) return -1;   // invalid case
    int prod = 1;
    while (n != 0)
    {
        prod *= n--;
    }
    return prod;
}

int main(int argc, char **argv)
{
    testing::InitGoogleTest(&argc, argv); // initializing it
    return RUN_ALL_TESTS();               // this will run all your tests
}
\end{Verbatim}

\newpage

\subsection{Test a function}
Now, let's test the \verb!factorial!function. Add the \verb!TEST! parts such
that
you final code looks
like following:
\begin{Verbatim}[frame=single]
#include <gtest/gtest.h>                // including the library
  
int factorial(int n)
{
    if (n < 0) return -1;   // invalid case
    int prod = 1;
    while (n != 0)
    {
        prod *= n--;
    }
    return prod;
}

TEST(FactorialTest, ZeroTest)  // for factorial of Zero
{
    ASSERT_EQ(1, factorial(0));
}
TEST(FactorialTest, PosTest)   // for factorial of Positive Numbers
{
    ASSERT_EQ(120, factorial(5));
    ASSERT_EQ(1, factorial(1));
    ASSERT_EQ(6, factorial(3));
}

int main(int argc, char **argv)
{
    testing::InitGoogleTest(&argc, argv); // initializing it
    return RUN_ALL_TESTS();               // this will run all your tests
}  
\end{Verbatim}

Compile and run to see the results.

\newpage

\subsection{Breaking it down}

Firstly, the general syntax of a \verb!TEST! looks like this:
\begin{Verbatim}[frame=single]
TEST(TestSuiteName, TestName)
{
    ... test body ...
}
\end{Verbatim}

Let's take one of the tests in our example and explain it line by line.

\begin{Verbatim}[frame=single]
TEST(FactorialTest, ZeroTest)  // for factorial of Zero
{
    ASSERT_EQ(1, factorial(0));
} 
\end{Verbatim}

First argument passed, \verb!FactorialTest!, is the
\texttt{TestSuiteName}, and the second argument, \verb!ZeroTest!,
is the \texttt{TestName} \\ 
A \textit{test suite} groups related tests. In our case, we collected all
tests related to the factorial function to a test suite called
\verb!FactorialTest!. 

You will populate each of your \textit{Test} with \textit{assertions}, which
are statements that will check whether a condition is true. A test is a failed
test if it crashes or if it has a failed assertion. In our case, we assert that
\texttt{factorial(0)} should be equal to 1. \\

Let's see the second test.
\begin{Verbatim}[frame=single]
TEST(FactorialTest, PosTest)   // for factorial of Positive Numbers
{
    ASSERT_EQ(120, factorial(5));
    ASSERT_EQ(1, factorial(1));
    ASSERT_EQ(6, factorial(3));
}  
\end{Verbatim}
Analyzing the header, this test belongs to the same test suite, \verb!FactorialTest!, as the previous test.
\\ \\
Not surprisingly, a test can have multiple assertions. If any one of those assertion fails, the test
\verb!PosTest! fails as well. \\

\subsection{ASSERT vs. EXPECT}
You can also replace the \verb!ASSERT_*! with \verb!EXPECT_*!,however, there will be a slight difference in
their performance. \verb!ASSERT_*! will immediately abort the current test if any one of the \verb!ASSERT_*!
statement fails. On the contrary, \verb!EXPECT_*! keeps on running even if one of the \verb!EXPECT_*! fails.
In our case, if \verb!ASSERT_EQ(120, factorial(5))! fails, two assertion statements following this statement
will not run, but they will run in the case of \verb!EXPECT_*!. \newpage



