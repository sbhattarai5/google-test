\section{Wrapping Up}

Let's wrap up with some extra information. \\

You should not be surprised to see different version of
\verb!ASSERT_*! and \verb!EXPECT_*!. The table below mentions some others: \\ \\
\begin{tabular}{| p{5.5cm} | p{5.5cm} | p{6cm} |}
\hline
\verb!ASSERT_*! & \verb!EXPECT_*! & VERIFIES \\
\hline \hline
\verb!ASSERT_EQ(val1, val2)! & \verb!EXPECT_EQ(val1, val2)! & \verb!val1 == val2! \\ 
\hline
\verb!ASSERT_NE(val1, val2)! & \verb!EXPECT_NE(val1, val2)! & \texttt{val1 != val2} \\
\hline
\verb!ASSERT_LT(val1, val2)! & \verb!EXPECT_LT(val1, val2)! & \verb!val1 < val2! \\
\hline
\verb!ASSERT_GT(val1, val2)! & \verb!EXPECT_GT(val1, val2)! & \verb!val1 > val2! \\
\hline
\verb!ASSERT_STREQ(val1, val2)! & \verb!EXPECT_STREQ(val1, val2)! & the two C strings have the same content \\
\hline
\end{tabular}
\\

This is all I have to say for Google Test. Of course, there are a lot
of other features like parameterized testing, typed testing which you
can learn more about. For more details, see the documentation for
Google Test. \\
(General documentation: \href{https://github.com/google/googletest/blob/master/googletest/docs/primer.md}{Link}) \\
(Advanced documentation: \href{https://github.com/google/googletest/blob/master/googletest/docs/advanced.md}{Link})


\newpage
