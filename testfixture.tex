\section{Test Fixture}

In this section, we will talk about \emph{Test Fixture}. As I have already
mentioned, this is useful when we use same data configuration of objects for
multiple tests.

\subsection{Let's create and test a Singly Linked List class}
Copy and paste the following code in a file called \verb!linkedlist.cpp!:
\begin{Verbatim}[frame=single]
#include <gtest/gtest.h> // including the library

// everything is made public to make this easy

template <typename T>
class Node      
{
public:
    Node(T val, Node< T > *next=NULL)
        : val_(val), next_(next)
    {}
    T val_;
    Node< T > *next_;
};

template <typename T>
class LinkedList
{
public:
    LinkedList()
        : head_(NULL)
    {}
    ~LinkedList()
    {
        while (head_ != NULL)
        {
            Node < T > * head = head_->next_;
            delete head_;
            head_ = head;
        }
    }


    void print() const
    {
        Node< T > *cptr = head_;
        while (cptr != NULL)
        {
            std::cout << cptr->val_ << ' ';
            cptr = cptr->next_;
        }
        std::cout << std::endl;
    }
    void addtohead(T val)
    {
        head_ = new Node< T >(val, head_);
    }
    Node< T > * deletehead() // returns pointer to the deleted node
    {
        if (head_ == NULL) return NULL;
        Node< T > * head = head_;
        head_ = head_ -> next_;
        return head;
    }
    Node< T > *head_;
};

int main(int argc, char **argv)
{
    testing::InitGoogleTest(&argc, argv); // initializing it
    return RUN_ALL_TESTS(); // this will run all your tests
}
\end{Verbatim}

It should be clear that to test a \emph{non-empty linked list} we need to create a linked list and add elements to it, and then test if the member functions work correctly. \\
Testing \verb!addtohead! in a non-empty list:
\begin{Verbatim}[frame=single]
TEST(LinkedListTest, AddToNonEmptyListTest)
{
    LinkedList< int > ll;
    ll.addtohead(5);
    ASSERT_NE(ll.head_, nullptr);
    EXPECT_EQ(5, ll.head_->val_);
}
\end{Verbatim}

\subsection{Example of ASSERT vs. EXPECT}
Notice that I used \verb!ASSERT! for checking if the head is \verb!nullptr!,
because there is no point on continuing if the head is \verb!nullptr!, because
you'll be dereferencing a \verb!nullptr! later. \\
Testing \verb!deletehead! from a non-empty list:
\begin{Verbatim}[frame=single]
TEST(LinkedListTest, DeleteFromNonEmptyListTest)
{
    LinkedList< int > ll;
    ll.addtohead(5);
    Node< int > *n = ll.deletehead();
    ASSERT_NE(n, nullptr);
    EXPECT_EQ(n->val_, 5);
    delete n;
}
\end{Verbatim}

However, the customization can be collected up and simplified. See next section.


\subsection{Creating a test fixture class}
We need to create a test fixture class and make our customized objects as the
member of that class. To create a test fixture clas, we derive a class from \verb!::testing::Test!. \\
Let's look at an example. For testing non-empty linked list, like our example in last section, we will do this:
\begin{Verbatim}[frame=single]
class NonEmptyLinkedListTest : public ::testing::Test
{
public:
    NonEmptyLinkedListTest()
      : ll1(new LinkedList< int >)
    {
        ll1->addtohead(5);
    }
    ~NonEmptyLinkedListTest()
    {
        delete ll1;
    }
    LinkedList< int > * ll1;
};
\end{Verbatim}

\newpage
\subsection{Syntax and usage}
Now, syntax for creating a test using test fixture looks like the following:

\begin{Verbatim}[frame=single]
TEST_F(TestFixtureName, TestName)
{
    ... test body ...
}  
\end{Verbatim}
\emph{Note: }The TestFixtureName argument should be same as the Test Fixture
class name that we define for customizing object.

In our case, our test looks like:
\begin{Verbatim}[frame=single]
TEST_F(NonEmptyLinkedListTest, AddToHeadTest)
{
    ASSERT_NE(ll1->head_, nullptr);
    EXPECT_EQ(5, ll1->head_->val_);
}
\end{Verbatim}
Using \texttt{TEST\textunderscore F()} will automatically create the text
fixture object for you, therefore, you can directly refer to the members in it.
\\ \\
For testing \texttt{deletehead},
\begin{Verbatim}[frame=single]
TEST_F(NonEmptyLinkedListTest, DeleteHeadTest)
{
    Node< int > *n = ll1->deletehead();
    ASSERT_NE(n, nullptr);
    EXPECT_EQ(n->val_, 5);
    delete n;
}
\end{Verbatim}

\emph{Note: }For each tests define with \texttt{TEST\textunderscore F()},
google test creates a fresh test fixture object. In our case, an object of
\texttt{NonEmptyLinkedLinkedTest} was freshly created for both
\texttt{AddToHeadTest} and \texttt{DeleteHeadTest}. \\

Furthermore, we can also define any methods inside test fixture class we want our tests to use. \\

\newpage


